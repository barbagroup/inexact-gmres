%% Manuscript for submission to SIAM 
\documentclass[final,leqno,]{siamltex1213}
% options that require additional .clo file:  
% onetabnum: to number equations consecutively with a single digit, instead of using the section number
% onetabnum: numbers tables consecutively throughout the paper
% onefignum: numbers figures consecutively throughout the paper

\usepackage[text={6in,8in},centering]{geometry} % load first
\usepackage{amsmath}
\usepackage{graphicx}
\usepackage{subfig}
\usepackage{xspace}

\graphicspath{{figs/}} %  PATH to figure files-- change to ./ for submission

% CUSTOM COMMAND DEFINITIONS
\newcommand{\R}{\mathbb{R}}
\newcommand{\Z}{\mathbb{Z}}
\newcommand{\K}{\mathbb{K}}
\newcommand{\cpu}{\textsc{cpu}}
\newcommand{\gpu}{\textsc{gpu}}

\newcommand{\bem}{\textsc{bem}\xspace}
\newcommand{\fmm}{\textsc{fmm}\xspace}
\newcommand{\fmmbem}{\fmm-\bem}

\newcommand{\cpp}{\textsc{c++}}
\newcommand{\blas}{\textsc{blas}}
\newcommand{\sse}{\textsc{sse}}

% CURLY LETTERS
\newcommand{\bigO}{\mathcal{O}}
\renewcommand{\O}[1]{\mathcal{O}(#1)}

% FMM OPERATOR DEFINITIONS
\newcommand{\ptom}{\textsc{p}\texttwooldstyle\textsc{m}\xspace} % P2M
\newcommand{\ltop}{\textsc{l}\texttwooldstyle\textsc{p}\xspace} % L2P
\newcommand{\mtop}{\textsc{m}\texttwooldstyle\textsc{p}\xspace} % M2P
\newcommand{\mtom}{\textsc{m}\texttwooldstyle\textsc{m}\xspace} % M2M
\newcommand{\mtol}{\textsc{m}\texttwooldstyle\textsc{l}\xspace} % M2L
\newcommand{\ltol}{\textsc{l}\texttwooldstyle\textsc{l}\xspace}  % L2L
\newcommand{\ptop}{\textsc{p}\texttwooldstyle\textsc{p}\xspace} % P2P

% MISC THINGS
\newcommand{\ncrit}{N_{\text{CRIT}}}
\newcommand{\pmin}{p_{\text{min}}}
\newcommand{\tsolve}{t_{\text{solve}}}


% SOLVER DEFINITIONS
\newcommand{\cg}{\textsc{cg}}
\newcommand{\gmres}{\textsc{gmres}\xspace}
\newcommand{\fgmres}{\textsc{fgmres}\xspace}
\newcommand{\bicgstab}{\textsc{bicgstab}\xspace}

% the text 'd' for integrals
\newcommand{\di}[1]{\text{d}#1}
% partial derivatives (frac)
\newcommand{\partiald}[2]{\frac{\partial #1}{\partial #2}}
% partial derivatives (inline)
\newcommand{\partialdi}[2]{\partial #1 / \partial #2}
% \hat{n}
\newcommand{\nhat}{\hat{n}}
% define a vector - undertilde
%\newcommand{\vect}[1]{\utilde{#1}}
% - bold
\newcommand{\vect}[1]{\mathbf{#1}}
% curly L
\renewcommand{\L}{\mathcal{L}}
% curly D
\newcommand{\D}{\mathcal{D}}
% sign
\newcommand{\sign}{\text{sign}}
% basis vectors
\newcommand{\e}{\vect{e}}
% dyadic product
\newcommand{\dyad}[2]{#1 \otimes #2}







\title{INEXACT KRYLOV ITERATIONS AND RELAXATION STRATEGIES WITH FAST-MULTIPOLE BOUNDARY ELEMENT METHOD} 

\author{Lorena A. Barba\thanks{The George Washington University, Washington DC, 20056 
(\email{labarba@gwu.edu})}
\and Simon K. Layton\thanks{If you want to be first author, you have to write ...}}



\begin{document}
\maketitle
%\slugger{sisc}{xxxx}{xx}{x}{x--x}
%slugger should be set to mms, siap, sicomp, sicon, sidma, sima, simax, sinum, siopt, sisc, or sirev

\begin{abstract}
Abstract text here.
\end{abstract}

\begin{keywords}\end{keywords}

\begin{AMS}\end{AMS}


\pagestyle{myheadings}
\thispagestyle{plain}
\markboth{L. A. Barba and S. K. Layton}{INEXACT KRYLOV ITERATIONS AND RELAXATION STRATEGIES}

\section{Introduction}

Lorem ipsum ...


%% METHODS
\section{Methods for the integral solution of elliptic equations using inexact {\small GMRES}}

\subsection{Boundary-integral solution of the Laplace equation}

To write the Laplace equation, $\nabla^{2}\phi(\vect{x}) = 0$,  in its integral formulation, we use the classical procedure of multiplying by the Green's function and integrating, applying the divergence theorem of Gauss and the chain rule, then dealing with singularities by a limiting process. This results in
%
\begin{equation}\label{eqn:laplace_bem_final}
	\frac{1}{2}\phi + \int_{\Gamma} \phi\partiald{G}{\nhat}\;\di{\Gamma} = \int_{\Gamma}\partiald{\phi}{\nhat}G\;\di{\Gamma},
\end{equation}

\noindent where $G = 1/4\pi r$ is the free-space Green's function for the Laplace equation ($\nabla^{2}G = -\delta$),  $\partiald{\cdots}{\nhat}$ represents the partial derivative in the direction normal to the boundary surface, and the integrals are on the boundary $\Gamma$ of the domain. The boundary element method consists of discretizing the boundary into surface panels and enforcing Equation \eqref{eqn:laplace_bem_final} on a set of target points (collocation version). In its typical form, surface panels take a constant value $\phi_j$, and the surface integrals become sums over $N$ flat surface elements, $\Gamma_j$, resulting in the following discretized equation:
%
\begin{equation}
	\frac{1}{2}\phi_i = \sum_j^{N} \partiald{\phi_j}{\nhat_j}\;\int_{\Gamma}G_{ij}\di{\Gamma_j} - \sum_j^{N} \phi_j\int_{\Gamma}\partiald{G_{ij}}{\nhat_j}\;\di{\Gamma_j}.
\end{equation}

Either the values of the potential or its normal derivative on each panel could be known from boundary conditions, resulting in either first-kind or second-kind integral equations. Finding the remaining unknowns requires solving a system of linear equations $A\vect{x}=\vect{b}$, where the elements of the coefficient matrix are
%
\begin{equation}
	A_{ij} = 
	\begin{cases}
		\int_{\Gamma} G_{ij}\;\di{\Gamma_j}, & \phi\;\text{given on panel}\;j \\
		\int_{\Gamma} \partiald{G_{ij}}{\nhat_j}\;\di{\Gamma_j}, & \partiald{\phi}{\nhat}\;\text{given on panel } j
	\end{cases}
\end{equation}

\noindent
and $\vect{b}$ is formed with the known terms: e.g., if $\phi$ is given on panel $j$, then $\phi_j\int_{\Gamma_j}\partialdi{G_{ij}}{\nhat_j}\;\di{\Gamma_j}$ will appear in the term $b_i$ on the right-hand side of the linear system.

Inserting for $G_{ij}$ and $\partialdi{G_ij}{\nhat_j}$ in terms of $1/r$ and $\nhat_j\cdot\nabla(1/r)$ results in
%
\begin{eqnarray}
	\label{eqn:laplace_bem_G}\int_{\Gamma} G_{ij}\;\di{\Gamma_j} & = & \int_{\Gamma} \frac{1}{|\vect{x}_i-\vect{x}_j|} \;\di{\Gamma_j} \\ 
	\label{eqn:laplace_bem_dGdn}\int_{\Gamma} \partiald{G_{ij}}{\nhat_j}\;\di{\Gamma_j} & = & \int_{\Gamma}\frac{d\vect{x}\cdot\nhat_j}{|\vect{x}_i-\vect{x}_j|^{3}}\;\di{\Gamma_j}
\end{eqnarray}

The next step is to apply an appropriate numerical integration scheme in order to generate all the terms of the coefficient matrix.

\subsection{Boundary-integral solution of the Stokes equation}

\subsection{Numerical integration methods}

\subsection{Krylov subspace methods}

\subsection{Fast multipole method}

%$ RESULTS
\section{Results and discussion}

\subsection{Inexact {\small GMRES} for the solution of Laplace's equation}
To start, we looked at grid-convergence comparing with the analytical solution using a sphere with constant potential and charge on the surface: $\phi = \partialdi{\phi}{\nhat} = 1$. To make surface triangulations of a sphere with increasing refinement, we started with an 8-triangle closed surface, then split recursively each triangle into four smaller ones. Figure \ref{fig:glob_spheres} shows two example discretizations. We solved the boundary-element problem by collocation in both the first-kind and second-kind integral formulations, using a standard right-preconditioned \gmres with fast-multipole-accelerated matrix-vector products and the semi-analytical integrals for the singular terms. For the far-field approximations, we used spherical-harmonic expansions with the following parameters in the \fmm: $\theta_{\text{MAC}} = 0.5$, $p = 10$, and a tolerance of $10^{-6}$ in the iterative solver. 
Figure \ref{fig:laplaceconvergence} shows the resulting convergence for both first-kind and second-kind formulations of the boundary element method on a sphere. They display the expected orders of convergence for boundary element methods: slightly faster than $\O{1/\sqrt{N}}$ in the 1st-kind formulation and $\O{1/N}$ for the 2nd-kind formulation. This gives confidence on our \bem code, the singular/near-singular integral calculations, and the far-field approximation using the \fmm.




\begin{figure}[h]
\begin{center}
	\subfloat[][128 panels]{\includegraphics[natwidth=4.73in,natheight=3.94in,width=0.4\textwidth]{sphere128.pdf}\label{fig:sphere128}}\qquad
	\subfloat[][2048 panels]{\includegraphics[natwidth=4.73in,natheight=3.94in,width=0.4\textwidth]{sphere2048.pdf}\label{fig:sphere2048}}
	\caption{Triangular discretizations of a spherical surface.}
	\label{fig:glob_spheres}
\end{center}
\end{figure}
%
\begin{figure}[t]
\begin{center}
	\includegraphics[natwidth=3in,natheight=2in,width=0.5\textwidth]{LaplaceConvergence.pdf}
	\caption{Convergence of 1st-kind (open circles with solid line) and 2nd-kind (open circles with dotted line) solvers for the Laplace equation on a sphere, using a right-preconditioned \gmres with \fmm-accelerated matrix-vecctor products, with parameters: $\theta_{\text{MAC}} = 0.5$, $p=10$, and solver tolerance of $10^{-6}$ (no relaxation).}
	\label{fig:laplaceconvergence}
\end{center}
\end{figure}

Next, we looked at the following test to see how the residual changes as the \gmres iterations proceed and  what value of $p$ is required in the \fmm-accelerated mat-vecs to continue convergence. We discretized a sphere with $32,768$ surface triangles and solved a first-kind integral equation using a solver tolerance of $10^{-5}$ with an initial value of $p$ set to 8. As the residual gets smaller, the value of $p$ needed to maintain convergence of the solver drops, and a low-$p$ of just 3 is sufficient by the seventh iteration. This offers the potential for substantial speed-ups in the calculations, because the translation operators of the \fmm scale from $\bigO(p^{4})$ for spherical harmonics to $\bigO(p^{6})$ for Cartesian expansions.
But we note that only the far-field evaluation can be sped-up with the relaxation strategy, which means that the correct balance between near field and far field in the \fmm could change as we reset $p$ in the later iterations.

\begin{figure}[h]
	\centering
	\includegraphics[natwidth=3.7in,natheight=2in,width=0.65\textwidth]{LaplaceResidualIterations.pdf}
	\caption{In a test using a sphere discretized with $32,768$ triangles, the residual $\|r_{k}\|$  (solid line, left axis) decreases with successive \gmres iterations while the necessary $p$ (open circles, right axis) to achieve convergence drops quickly.}
	\label{fig:residualp}
\end{figure}

To find out how much is the potential speed-up, we compared the time to solution for different cases with and without the relaxation strategy. Using three surface discretizations, we solved the boundary-element problem with 1st- and 2nd-kind formulations to a solver tolerance of $10^{-5}$, using a multi-threaded evaluator on 4 \cpu\ cores. In each case, we were careful to set the value of $\ncrit$ to minimize the time to solution of the particular test case. Figure \ref{fig:relaxation_timing} shows the speed-up in the time spent solving the linear system of equations to the specified tolerance. The detailed results are given in the subsequent Tables.


\begin{figure}%[h]
	\centering
	\includegraphics[natwidth=3in,natheight=2in,width=0.45\textwidth]{LaplaceSpeedupRelaxation.pdf}
	\caption{Speedup using a relaxation strategy for three different triangulations of a sphere, using 1st-kind and 2nd-kind integral formulations. (Multi-threaded evaluator running on 4 \cpu\ cores.)}
	\label{fig:relaxation_timing}
\end{figure}


\begin{table}[h]
\footnotesize
\begin{center}
\begin{tabular}{c|cc|cc|c}
  & \multicolumn{2}{c|}{Non-Relaxed} & \multicolumn{2}{c|}{Relaxed} & \\
  N & $\ncrit$ & $\tsolve$ & $\ncrit$ & $\tsolve$ & Speed-up \\
 \hline
   & & & & & \\
  2048 & 400 & 0.27 & 200 & 0.40 & 0.675 \\
  8192 & 400 & 2.58 & 200 & 1.7 & 1.52 \\
  32768  & 400 & 7.1 & 200 & 5.07 & 1.40 \\
  131072  & 400 & 30.1 & 200 & 20.72 & 1.45 \\
 
\end{tabular}
\end{center}
\caption{Speed-ups for the relaxation strategy on a Laplace 1st-kind integral solver, $p=8$, solver tolerance $10^{-5}$.}
\label{tab:laplace_1st_relaxation}
\end{table}%

\begin{table}[h]
\footnotesize
\begin{center}
\begin{tabular}{c|cc|cc|c}
  & \multicolumn{2}{c|}{Non-Relaxed} & \multicolumn{2}{c|}{Relaxed} & \\
  N & $\ncrit$ & $\tsolve$ & $\ncrit$ & $\tsolve$ & Speed-up \\
 \hline
   & & & & & \\
  2048 & 400 & 0.13 & 200 & 0.64 & 0.20 \\
  8192 & 400 & 1.49 & 200 & 1.19 & 1.25 \\
  32768 & 400 & 6.77 & 200 & 5.92 & 1.14 \\
  131072 & 400 & 30.01 & 200 & 24.2 & 1.24 \\
 
\end{tabular}
\end{center}
\caption{Speed-ups for the relaxation strategy on a Laplace 2nd-kind integral solver, $p=8$, solver tolerance  $10^{-5}$.}
\label{tab:laplace_2nd_relaxation}
\end{table}%

The results on Tables \ref{tab:laplace_1st_relaxation} and \ref{tab:laplace_2nd_relaxation} show a speed-up for the three larger grids---also plotted on Figure \ref{fig:relaxation_timing}---of about $1.4\times$ for the 1st-kind integral formulation and $1.2\times$ for the 2nd-kind formulation. These are moderate speedups, but these tests already taught us something: that one has to give up on the idea of partitioning the domain between a near-field and a far-field in a way that balances the time spent computing each one---an accepted idea in \fmm applications. When relaxing the accuracy of the \gmres iterations, the time taken to compute the far field decreases significantly. This means that to minimize time-to-solution when using relaxed \gmres, the near and far fields should not be balanced, but rather the far field should be bloated. As a result, the first couple of iterations are completely dominated by the time to compute the far field, but this is offset by the benefit of much cheaper iterations from then on. This is a simple but unexpected and counter-intuitive algorithmic consequence of the idea of inexact \gmres with \fmm.

It's clear that greater benefits from the relaxation strategy should derive from two situations: (a) where higher accuracy is needed (necessitating an initially higher $p$) , and (b) where the linear system demands a greater number of iterations to reach a set tolerance (resulting in more computations done at low $p$). To demonstrate this, we now present two tests that force these situations. 

\begin{figure}[h]
	\centering
	\includegraphics[natwidth=3in,natheight=2in,width=0.45\textwidth]{LaplaceRelaxationP.pdf}
	\caption{Timings for solving a 1st-kind Laplace integral formulation on a sphere discretized with $32,768$ panels, using a relaxed \gmres with different initial values of $p$, compared with a fixed-$p$ solver. The iteration count was capped at 10 for all cases. (Multi-threaded evaluator running on 4 \cpu\ cores.)}
	\label{fig:laplace_p_speedup}
\end{figure}

Figure \ref{fig:laplace_p_speedup} shows the timings obtained in a situation where the initial $p$ is incrementally larger, representing applications that demand higher accuracy. The test consists of a 1st-kind Laplace integral solver on a sphere discretized with $32,768$ panels, enforcing a fixed number of $10$ \gmres iterations (which results in a residual of $~5.5\times 10^{-5}$ for the sphere). Table \ref{tab:laplace_1st_p_relaxation} shows the data for this test: the speed-up is greater with larger initial values of $p$, leveling at around $2\times$ when initial $p$ is greater than 8. Note again that the shortest time-to-solution requires an unbalanced tree on the first iteration, with a bloated far field. This means that the first (high-$p$) iteration is much slower than the corresponding fixed-$p$ case: the speed-up is thus purely a product of the low-cost later iterations.

\begin{table}[h]
\footnotesize
\begin{center}
\begin{tabular}{c|cc|cc|c}
  & \multicolumn{2}{c|}{Relaxed} & \multicolumn{2}{c|}{Non-Relaxed} & \\
  $p$ & $\ncrit$ & $\tsolve$ & $\ncrit$ & $\tsolve$ & Speedup \\
   \hline
   & & & & & \\
  5 & 100 & 4.21 & 100 & 6.34 & 1.51 \\
  8 & 100 & 6.24 & 400 & 12.4 & 1.99 \\
  10 & 150 & 8.76 & 400 & 18.5 & 2.11 \\
  12 & 150 & 13.2  & 600 & 25.3 & 1.92 \\
  15 & 150 & 19.3 & 600 & 38.3 & 1.98 \\
 
\end{tabular}
\end{center}
\caption{Speed-up when using a relaxation strategy on a Laplace 1st-kind integral solver, compared with a non-relaxed solver, with increasing value of the initial $p$ (representing increased accuracy demands of the application), for a sphere discretized with $32,768$ panels. (Multi-threaded evaluator running on 4 \cpu\ cores.)}
\label{tab:laplace_1st_p_relaxation}
\end{table}%

The final case looks at the situation where the application might demand large iteration counts, as could be encountered in harder-to-precondition cases. Keeping the value of $p$ fixed at 10, we ran several cases with increasing number of (set) \gmres iterations, representing more ill-conditioned problems. Figure \ref{fig:laplace_iterations_speedup} shows how cases with larger iteration counts experience greater speed-up from the relaxation strategy. As seen in the data of Table \ref{tab:laplace_1st_iterations_relaxation}, each non-relaxed iteration adds approximately $1.68$s to $\tsolve$, while each relaxed iteration adds $0.276$s; we can thus extrapolate a speed-up nearing $5\times$ at 100 iterations.



\begin{figure}[h]
	\centering
	\includegraphics[natwidth=3in,natheight=2in,width=0.5\textwidth]{LaplaceRelationIterations.pdf}
	\caption{Speed-ups for solving a 1st-kind Laplace integral problem on a sphere discretized with $32,768$ panels, as the \gmres iteration count increases; $p=10$ for all cases. (Multi-threaded evaluator running on 4 \cpu\ cores.)}
	\label{fig:laplace_iterations_speedup}
\end{figure}


\begin{table}[h]
\footnotesize
\begin{center}
\begin{tabular}{c|cc|cc|c}
  & \multicolumn{2}{c|}{Relaxed} & \multicolumn{2}{c|}{Non-Relaxed} & \\
  Iterations & $\ncrit$ & $\tsolve$ & $\ncrit$ & $\tsolve$ & Speedup \\
 \hline
   & & & & & \\
  5 & 100 & 8.57 & 400 & 10.0 & 1.17 \\
  10 & 100 & 9.81 & 400 & 18.4 & 1.88 \\
  15 & 100 & 11.1 & 400 & 26.9 & 2.42 \\
  20 & 100 & 12.4 & 400 & 35.3 & 2.85 \\
  25 & 100 & 13.7 & 400 & 43.8 & 3.20 \\
  50 & 100 & 20.6 & 400 & 86.2 & 4.18 \\
 
\end{tabular}
\end{center}
\caption{Speed-up of the relaxation strategy on solving a Laplace 1st-kind integral problem on a sphere discretized with $32,768$ panels, with increasing iteration count; $p$ fixed to a value of 10. (Multi-threaded evaluator running on 4 \cpu\ cores.)}
\label{tab:laplace_1st_iterations_relaxation}
\end{table}%


\section{Conclusion} 

We have shown the first successful application of a relaxation strategy for fast-multipole-accelerated boundary element methods. Testing the relaxation strategy on Laplace problems, we confirmed that it converges to the right solution, it provides moderate speed-ups over using a fixed $p$, and it leads to initially bloated far-fields to obtain the minimum time to solution.
Exploring the performance advantage of relaxing the value of $p$ as \gmres iterations advance, we conclude that problems requiring high accuracy and/or resulting in more ill-conditioned linear systems will experience the best speed-ups, which for Laplace problems can be in the order of $2-4\times$.


 
%% Acnowledgements

%% Bibliography


\end{document}
%% end of file `docultex.tex'
