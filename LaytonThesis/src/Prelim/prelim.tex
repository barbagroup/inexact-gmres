%!TEX root = ../thesis.tex

% This file contains all the necessary setup and commands to create
% the preliminary pages according to the buthesis.sty option.

\title{Fast multipole boundary element solutions with inexact Krylov iterations and relaxation strategies}

\author{Simon Layton}

% Type of document prepared for this degree:
%   1 = Master of Science thesis,
%   2 = Doctor of Philisophy dissertation.
%   3 = Master of Science thesis and Doctor of Philisophy dissertation.
\degree=2

\prevdegrees{B.Sc., University of Bristol, 2008 \\
M.S., Boston University, 2011}

\department{Department of Mechanical Engineering}

% Degree year is the year the diploma is expected, and defense year is
% the year the dissertation is written up and defended. Often, these
% will be the same, except for January graduation, when your defense
% will be in the fall of year X, and your graduation will be in
% January of year X+1
\defenseyear{2013}
\degreeyear{2013}

% For each reader, specify appropriate label {First, second, third},
% then name, then title. Warning: If you have more than five readers
% you are out of luck, because it will overflow to a new page.
% Sometimes you may wish to put part of the title in with the name
\reader{First}{Lorena Barba, PhD}
{Assistant Professor of Mechanical Engineering}
\reader{Second}{Sheryl Grace, PhD}{Associate Professor of Mechanical Engineering}
\reader{Third}{Paul Barbone, PhD}{Associate Professor of Mechanical Engineering}
\reader{Fourth}{Emily Ryan, PhD}{Assistant Professor of Mechanical Engineering}

% The Major Professor is the same as the first reader, but must be
% specified again for the abstract page
\majorprof{Lorena A. Barba, PhD}{\mbox{Department of Mechanical Engineering}}

%                       PRELIMINARY PAGES
% According to the BU guide the preliminary pages consist of:
% title, copyright (optional), approval,  acknowledgments (opt.),
% abstract, preface (opt.), Table of contents, List of tables (if
% any), List of illustrations (if any). The \tableofcontents,
% \listoffigures, and \listoftables commands can be used in the
% appropriate places. For other things like preface, do it manually
% with something like \newpage\section*{Preface}.

% This is an additional page (do not hand it in at the library) to print
% boxed-in title, author and degree statement so that they are visible through
% the opening in BU covers used for reports. This makes a nicely bound copy.
\buecethesistitleboxpage

% Make the titlepage based on the above information.  If you need
% something special and can't use the standard form, you can specify
% the exact text of the titlepage yourself.  Put it in a titlepage
% environment and leave blank lines where you want vertical space.
% The spaces will be adjusted to fill the entire page.
\maketitle
\cleardoublepage

% The copyright page is blank except for the notice at the bottom. You
% must provide your name in capitals.
\copyrightpage
\cleardoublepage

% Now include the approval page based on the readers information
\approvalpage
\cleardoublepage

% Here goes your favorite quote.
\newpage
\thispagestyle{empty}
%!TEX root = ../thesis.tex

\phantom{.}
\vspace{4in}

\begin{singlespace}
\begin{quote}
  %\textit{Facilis descensus Averni;}\\
  %\textit{Noctes atque dies patet atri janua Ditis;}\\*
  %\textit{Sed revocare gradum, superasque evadere ad auras,}\\
  %\textit{Hoc opus, hic labor est.}\hfill{Virgil (from Don's thesis!)}
  \textit{To my parents Dennis and Wendy, and my beloved fianc\'ee Elizabeth}
\end{quote}
\end{singlespace}

% \vspace{0.7in}
%
% \noindent
% [The descent to Avernus is easy; the gate of Pluto stands open night
% and day; but to retrace one's steps and return to the upper air, that
% is the toil, that the difficulty.]

\cleardoublepage

% The acknowledgment page should go here. Use something like
% \newpage\section*{Acknowledgments} followed by your text.
\newpage
\section*{\centerline{Acknowledgments}}
%!TEX root = ../thesis.tex

%I would like to thank Jonathan Polimeni for cleaning up all these old
%LaTeX style files and templates so that I didn't have to use Word to
%write up this document. Also, I would like to thank all the CV/CNS lab
%graduates for their contributions and tweaks to this organizational
%scheme over the years (after many dissapointing interactions with
%Martha Wellman).
%
%This brings me to thank Martha Wellman of CAS and Brendon McDermot of
%Mugar library who together uphold the stylistic and aesthetic
%conventions that have been implemented in this LaTeX manuscript. Also
%Sister Mary Virginia, the Thesis/dissertation coordinator of Mugar
%Library, deserves some thanks as well.
%
%Finally, I would like to thank Stephen Gildea, for the MIT sytle file
%off which this current version is based, and Paolo Gaudiano for
%porting the MIT style to one compatible for BU.

% [[[ LORENA ]]]
I would like to express wholehearted gratitude to my academic advisor over the last 6 years, Prof. Barba for her support and guidance throughout my studies in two universities, 3 degrees and 2 continents. In particular I would like to thank her for the belief she showed in me by bringing me from Bristol to Boston University in order to continue my studies. I deeply appreciate the freedom I have been given in my research, along with the invaluable advice and support that has been key to my successes.

% [[[ COMMITTEE ]]]
I would also like to thank the members of my committee, Professors Paul Barbone, Sheryl Grace and Emily Ryan for their time and insights during these final months of my studies.

% [[[ GROUP ]]]
With the contributions, ideas and friendship of Christopher Cooper, Anush Krishnan, Brent Parham and Olivier Mesnard in Prof. Barba's research group, the time I have spent here has flown by, and they have always been willing to help in any way necessary, be it debugging code, reading papers or spending time relaxing away from the office. Special thanks must also go to Rio Yokota, who was always available for advice on coding, {\fmm} or any other subject.

% [[[ OTHERS ]]]
I also take this opportunity to thank those who have helped and funded me through my time at Boston University -- Dan Kamalic for his patience in helping me with the use of his compute clusters. My continued funding has been greatly appreciated, and I would like to thank Boston University, the Massachusetts Green High-Performance Computing Center (MGHPCC), National Science Foundation (NSF) and the Office of Naval Research (ONR) for their contributions. 

% [[[ PARENTS ]]]
I would like to thank my fianc\'ee, Elizabeth Matos for the love and support she has shown me, both pulling me through the last parts of my studies, and giving me an incredible future to look forward to. Finally, I wish to show my appreciation for my parents, Dennis and Wendy Layton for their unending and unconditional encouragement and belief, for always being there for me, and for not begrudging me this time far away from them in Boston. 

\cleardoublepage

% The abstractpage environment sets up everything on the page except
% the text itself.  The title and other header material are put at the
% top of the page, and the supervisors are listed at the bottom.  A
% new page is begun both before and after.  Of course, an abstract may
% be more than one page itself.  If you need more control over the
% format of the page, you can use the abstract environment, which puts
% the word "Abstract" at the beginning and single spaces its text.

\begin{abstractpage}
%!TEX root = ../thesis.tex

Boundary element methods (\bem) have been used for years to solve a multitude of engineering problems, ranging from Bioelectrostatics, to fluid flows over micro-electromechanical devices and deformations of cell membranes. Only requiring the discretization of a surface into panels rather than the entire domain, they effectively reduce the dimensionality of a problem by one.

This reduction in dimensionality nevertheless comes at a cost. {\bem} requires the solution of a large, dense linear system with each matrix element formed of an integral between two panels, often performed used an iterative solver based on Krylov subspace methods. This requires the repeated calculation of a matrix vector product that can be approximated using a hierarchical approximation known as the fast multipole method (\fmm). While adding complexity, this reduces order of the time-to-solution from $\O{cN^{2}}$ to $\O{cN}$, where $c$ is some function of the condition number of the dense matrix.

This thesis obtains algorithmic speedups for the solutions of {\fmmbem} systems by applying the mathematical theory behind inexact matrix-vector products to our solver, implementing a relaxation scheme to control the error incurred by the {\fmm} in order to minimize the total time-to-solution. The theory is extensively verified for both Laplace equation and Stokes flow problems, with an investigation to determine how further problems may benefit from the addition of a relaxed solver. We also present experiments for the Stokes flow around both single and multiple red blood cells, an area of ongoing research, showing good speedups that would be applicable for any other code that chose to implement a similar relaxed solver. All of these results are obtained with an easy-to-use, extensible and open-source {\fmmbem} code.


% Engineering problems become ever larger, and we continually look for better ways to simulate them. Boundary elements (BEM) are a integral formulation over a surface, only requiring discretization of the surface rather than the entire domain, effectively reducing the dimensionality of a problem from a volume to a surface in 3D.

% BEM requires the iterative solution of a large dense linear system, in our case solved using GMRES, a Krylov subspace method that uses repeated matrix-vector products. These are accelerated using a fast multipole method or treecode, reducing the time to solution from O(N^2) to O(N).

% We apply the mathematical theory behind inexact matrix products and relaxation to our BEM solver, utilizing the ability of fast multipole methods to control error in order to minimize time to solution. This is combined with preconditioners and inner-outer solvers, and extensively tested in a new, extensible computational framework.

\end{abstractpage}
\cleardoublepage

% Now you can include a preface. Again, use something like
% \newpage\section*{Preface} followed by your text

% Table of contents comes after preface
\tableofcontents
\cleardoublepage

% If you have tables, uncomment the following line
%\listoftables
%\cleardoublepage

% If you have figures, uncomment the following line
\newpage
\listoffigures
\cleardoublepage

% List of Abbrevs is NOT optional (Martha Wellman likes all abbrevs listed)
\chapter*{List of Abbreviations}
\begin{center}
  \begin{tabular}{lll}
    \hspace*{2em} & \hspace*{1in} & \hspace*{4.5in} \\
    BEM  & \dotfill & Boundary Element Method \\
    FMM  & \dotfill & Fast Multipole Method \\
    GMRES & \dotfill & Generalized minimal residual \\
    FGMRES & \dotfill & Flexible generalized minimal residual \\
    BLAS & \dotfill & Basic Linear Algebra Subroutines \\
    P2P & \dotfill & Particle-to-Particle \\
    P2M & \dotfill & Particle-to-Multipole \\
    M2M & \dotfill & Multipole-to-Multipole \\
    M2L & \dotfill & Multipole-to-Local \\
    L2L & \dotfill & Local-to-Local \\
    L2P & \dotfill & Local-to-Particle
  \end{tabular}
\end{center}
\cleardoublepage

% List of algorithms
\newpage
\listofalgorithms
\cleardoublepage

% END OF THE PRELIMINARY PAGES

\newpage
\endofprelim
